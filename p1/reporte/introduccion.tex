\section{Introducción}
Este reporte es sobre tres distintas arquitecturas de redes neuronales, las cuales son: Hamming Perceptron Simple y ADALINE. Se incluye una breve explicación sobre estas arquitecturas el como funcionan y sus diferentes representaciones para que sea fácil de entender.
\\\\
Además, se programaron estas tres arquitecturas, sujetas a algunas restricciones para que fuera más sencilla su elaboración, el desarrollo de estos programas fue realizado en \emph{MATLAB} ya que nos proporciona un manejo sencillo de matrices, lo cual es el principal elemento con el que trabajan estas redes neuronales. 
\\\\
Para verificar el correcto funcionamiento de los programas se realizaron pruebas con entradas de diferentes tamaños y valores, de igual forma se muestra en pantalla el resultado y una representación gráfica. Para complementar los resultados obtenidos se incluye un análisis de estos.
\\\\
Finalmente, tenemos una sección de conclusiones sobre la parte más importante de este reporte.