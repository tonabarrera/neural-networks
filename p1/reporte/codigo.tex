    \section{Anexo}
        En esta sección se encuentra el código de los tres programas desarrollados en MATLAB.
        \subsection{Hamming}
        \begin{lstlisting}
% Cada elemento del vector de entrada tiene solo dos posibles valores
opcion = input('Ingresa el nombre del archivo: ', 's');
archivo = dlmread(opcion);
tam = size(archivo);
% Tam de nuestros vectores prototipo
R = tam(2);

% Numero de neuronas, corresponde a cada vector prototipo
S = tam(1) - 1;

% Vector a clasificar
p = archivo(S+1, :)';

% Las filas de W1 son los vectores prototipos
% Inicializacion de W1
W1 = archivo(1:S, :);

% Cada elemento del bias es el tam del vector prototipo
% Inicializacion del bias
b1 = ones(S, 1) * R;

% Propagamos hacia adelante
a1 = purelin((W1*p)+b1);
% Fin de la capa feedforward

%Inicio de la capa recurrente
a2 = a1;
% Obtencion del valor epsilon 0 < epsilon < 1/(S-1)
epsilon = round(rand(1)*1/(S-1), 4); 
% Inicializacion y llenado de la W2 de la capa recurrente
W2 = ones(S, S);
for i = 1:S
    for j = 1:S
        if i==j
            W2(i, j) = 1;
        else
            W2(i, j) = -epsilon; 
        end;
    end;
end;

% Aqui se guardara la salida de la capa recurrente
% Metomos la salida de la capa anterior
salida = fopen('salida_hamming.txt', 'w');
fprintf(salida, '%.15f ', a2);
fprintf(salida, '\n');
% Recurrencia de la capa
t = 1;
while true
    % Obtenemos el t+1
    a2_sig = poslin(W2*a2);
    fprintf(salida, '%.15f ', a2_sig);
    fprintf(salida, '\n');
    if a2_sig == a2
        % Si ya terminamos detenemos el ciclo
        fclose(salida);
        break;
    else
        % Siguiente iteracion
        a2 = a2_sig;
    end;
    t = t + 1;
end;

% Fin de la capa recurrente
fprintf('Termino en la iteracion %d\n', t);
v = 1;
for ite = a2'
    if ite ~= 0
        break;
    else
        v = v+1;
    end;
end
fprintf('La clase a la que convergio fue: %d\n', v);
% Imprimir datos y graficar la salida de a2
a2_recurrente = dlmread('salida_hamming.txt');
figure('Name', 'Evolucion de la salida de la capa recurrente');
plot(0:t, a2_recurrente, 'LineWidth', 2);
hold;
grid;
xlabel('t');
ylabel('a^2(t)');
etiquetas = cell(1, S);
for i = 1:S
    etiquetas{i} = ['P_' num2str(i)];
end;
legend(etiquetas);
        \end{lstlisting}
        \subsection{Perceptron}
        \subsection{ADALINE}
        \begin{lstlisting}
%% Funcion principal
function adaline()
    opcion = input('1.-Red con bias   2.-Red sin bias: ', 's');
    if str2double(opcion) == 1
        adaline_bias();
    else
        % Captura de los datos
        tam = input('Dame el tam del codificador: ', 's');
        tam = str2double(tam);
        tabla_verdad = zeros(2^tam, tam+1);
        eit = input('Dame el eit: ', 's');
        eit = str2double(eit);
        alpha = input('Dame el valor de alpha: ', 's'); % 0.3
        alpha = str2double(alpha);
        N = 2^tam-1;
        W = rand(1, tam);
        auxiliar_w = fopen('auxiliar_w.txt', 'w');
        auxiliar_Eit = fopen('auxiliar_Eit.txt', 'w');
        fprintf(auxiliar_w, '%.10f ', W);
        fprintf(auxiliar_w, '\n');
        % Llenamos nuestra tabla de verdad
        for i = 0:N
            a = dec2bin(i, tam) - '0';
            tabla_verdad(i+1, 1:tam) = a;
            tabla_verdad(i+1, tam+1) = i;
        end;
        continuar = true;
        iteracion = 1;
        while continuar
            Eit = 0;
            for n = 1:N
                p = tabla_verdad(n, 1:tam)';
                a = purelin(W*p);
                t = tabla_verdad(n, tam+1);
                ed = t-a;
                Eit = Eit + ed;
                W = W + (2 * alpha * ed*p');
            end
            Eit = 1/N * Eit;
            Eit = abs(Eit);
            fprintf(auxiliar_Eit, '%.10f ', Eit);
            fprintf(auxiliar_Eit, '\n');
        
            fprintf(auxiliar_w, '%.10f ', W);
            fprintf(auxiliar_w, '\n');
            if Eit == 0
                disp('Criterio de igualdad a cero');
                break;
            elseif Eit < eit
                disp('Criterio de menor que el error');
                break;
            end
            iteracion = iteracion + 1;
        end
        fclose(auxiliar_Eit);
        fclose(auxiliar_w);
    
        % Desplegar los valores finales
        disp('Valores finales de W');
        disp(W);
        
        % Figura de los valores de W
        valoresW = dlmread('auxiliar_w.txt');
        graficar_pesos(tam, valoresW, iteracion);
        % Final de la grafica de W
        
        % Otra figura para mostrar en otra ventana
        valoresEit = dlmread('auxiliar_Eit.txt');
        graficar_error(iteracion, valoresEit)
        % Final de la grafica de error
        
        % Guardamos en un archivo
        a_final = strcat('resultado_', datestr(now, 'HH-MM_dd-mm-yyyy'));
        a_final = strcat(a_final, '.txt');
        valores_finales = fopen(a_final, 'w');
        fprintf(valores_finales, 'Valores finales de W \n');
        fprintf(valores_finales, '%.10f ', W);
        fprintf(valores_finales, '\n');
        fclose(valores_finales);
    end
end % Final de la funcion principal

%% graficar_pesos: function description
function graficar_pesos(tam, valoresW, iteracion)
    figure('Name', 'Evolucion de los pesos');
    % Grafica un vector en x y otro vector en y
    plot(0:iteracion, valoresW, 'LineWidth', 2); 
    hold;
    grid;
    % Etiquetas de los ejes
    xlabel('Iteracion');
    ylabel('W');
    
    % Titulo de nuestra grafica
    etiquetas = cell(1, tam);
    for i = 1:tam
        etiquetas{i} = ['W_' num2str(i)];
    end;
    legend(etiquetas);
    title('Valores de W');
end

%% graficar_error: function description
function graficar_error(iteracion, valoresEit)
    figure('Name', 'Error Eit');
    x = 1:iteracion;
    % Grafica un vector en x y otro vector en y
    plot(x, valoresEit, 'LineWidth', 2);
    hold;
    plot(x, valoresEit, '*', 'LineWidth', 2);
    grid;
    % Imprime las coordenadas de Eit
    strValues = strtrim(cellstr(num2str([x(:) valoresEit(:)], '(%d,%d)')));
    text(x, valoresEit, strValues, 'VerticalAlignment', 'bottom');
    % Etiquetas de los ejes
    xlabel('Iteracion');
    ylabel('E_{it}');
    % Titulo de nuestra grafica
    title('Valores de E_{it}');
end

%Con bias
function adaline_bias()
    archivo = input('Dame el nombre del archivo: ', 's');
    prueba = fopen(archivo, 'r');
    S = 0;
    targets = [];
    prototipos = [];
    R = 0;
    dimen = [];
    tipo_lectura = 0;
    while feof(prueba) == 0
        linea = fgetl(prueba);
        if linea ~= '{'
            fclose(prueba);
            datos = dlmread(archivo);
            tam = size(datos);
            S = 1;
            prototipos = datos(:, 1:tam(2)-1)';
            dimen = size(prototipos);
            targets = datos(:, tam(2))';
            R = dimen(1);
            tipo_lectura = 1;
            break;
        else
            linea = linea(2:length(linea)-1);
            proto = linea(2:find(linea==',')-2);
            tar = linea(find(linea==',')+2:length(linea)-1);
            proto = str2num(proto);
            tar = str2num(tar);
            prototipos = [prototipos proto'];
            targets = [targets tar'];
        end
    end
    if tipo_lectura == 0
        S = 2;
        dimen = size(prototipos);
        R = dimen(1);
    end   
    itmax = input('Ingrese valor de itmax: ', 's'); %5
    itmax = str2double(itmax);
    alpha = input('Dame el valor de alpha: ', 's'); % 0.3
    alpha = str2double(alpha);
    eit = input('Dame el eit: ', 's');
    eit = str2double(eit);
    W = ones(S, R);
    b = ones(S, 1);
    
    auxiliar_w = fopen('auxiliar_w.txt', 'w');
    auxiliar_bias = fopen('auxiliar_bias.txt', 'w');
    auxiliar_error = fopen('auxiliar_Eit.txt', 'w');
    fprintf(auxiliar_w, '%.10f ', W);
    fprintf(auxiliar_w, '\n');
    
    fprintf(auxiliar_bias, '%.10f ', b);
    fprintf(auxiliar_bias, '\n');
    criterio = 0;
    for iteracion = 1:itmax
        Eit = 0;
        for n = 1:dimen(2)
            p = prototipos(:, n);
            a = purelin(W*p + b);
            t = targets(:, n);
            ed = t-a;
            Eit = Eit + ed;
            W = W + (2 * alpha * ed * p');
            b = b + (2 * alpha * ed);
        end
        Eit = 1/dimen(2) * Eit;
        Eit = abs(Eit);
        fprintf(auxiliar_error, '%.10f ', Eit);
        fprintf(auxiliar_error, '\n');
        
        fprintf(auxiliar_w, '%.10f ', W);
        fprintf(auxiliar_w, '\n');
        
        fprintf(auxiliar_bias, '%.10f ', b);
        fprintf(auxiliar_bias, '\n');
        if Eit == 0
            criterio = 1;
            break;
        elseif Eit < eit
            criterio = 2;
            break;
        end
    end
    fclose(auxiliar_error);
    fclose(auxiliar_w);
    fclose(auxiliar_bias);
    
    % Figura de los valores de W
    valoresW = dlmread('auxiliar_w.txt');
    valores_bias = dlmread('auxiliar_bias.txt');
    graficar_pesos_bias(valoresW, iteracion, S, R);
    graficar_bias(valores_bias, iteracion, S);
    % Final de la grafica de W
    
    % Otra figura para mostrar en otra ventana
    valoresEit = dlmread('auxiliar_Eit.txt');
    graficar_error_bias(iteracion, valoresEit, S);
    % Final de la grafica de error
    
    if criterio == 0
        disp('Termino alcanzando el maximo de iteraciones')
    elseif criterio == 1
        disp('Termino por criterio de error igual a 0');
        fprintf('Termino en la iteracion %d\n', iteracion);
        disp('Valores finales de W:');
        disp(W);
        disp('Valores finales del bias:');
        disp(b);
    else
        disp('Termino por criterio de menor al error permitido');
        fprintf('Termino en la iteracion %d\n', iteracion);
        disp('Valores finales de W:');
        disp(W);
        disp('Valores finales del bias:');
        disp(b);
    end
    
    archivo_final = strcat('resultado_', datestr(now, 'HH-MM_dd-mm-yyyy'));
    archivo_final = strcat(archivo_final, '.txt');
    finales = fopen(archivo_final, 'w');
    fprintf(finales, 'Valores finales de W: \n');
    fprintf(finales, '%.10f ', W);
    fprintf(finales, '\n');
    
    fprintf(finales, 'Valores finales del bias: \n');
    fprintf(finales, '%.10f ', b);
    fprintf(finales, '\n');
    fclose(finales);
end


%% graficar_pesos: function description
function graficar_pesos_bias(valoresW, iteracion, S, R)
    figure('Name', 'Evolucion de los pesos');
    % Grafica un vector en x y otro vector en y
    plot(0:iteracion, valoresW, 'LineWidth', 2); 
    hold;
    grid;
    % Etiquetas de los ejes
    xlabel('Iteracion');
    ylabel('W');
    
    etiquetas = cell(1, S*R);
    k = 1;
    for i = 1:S
        for j = 1:R
            etiquetas{k} = ['W_{' num2str(i) ',' num2str(j) '}'];
            k = k+1;
        end;
    end;
    legend(etiquetas);
    title('Valores de W');
end

%% graficar_error: function description
function graficar_error_bias(iteracion, valoresEit, S)
    figure('Name', 'Error Eit');
    x = 1:iteracion;
    % Grafica un vector en x y otro vector en y
    plot(x, valoresEit, 'LineWidth', 2);
    hold;
    grid;
    % Etiquetas de los ejes
    xlabel('Iteracion');
    ylabel('E_{it}');
    % Titulo de nuestra grafica
    etiquetas = cell(1, S);
    for i = 1:S
        etiquetas{i} = ['Neurona ' num2str(i)];
    end;
    legend(etiquetas);
    % Titulo de nuestra grafica
    title('Valores de E_{it}');
end

%% graficar_bias
function graficar_bias(valores_bias, iteracion, S)
    figure('Name', 'Evolucion del bias');
    % Grafica un vector en x y otro vector en y
    plot(0:iteracion, valores_bias, 'LineWidth', 2); 
    hold;
    grid;
    % Etiquetas de los ejes
    xlabel('Iteracion');
    ylabel('Bias');
    % Titulo de nuestra grafica
    etiquetas = cell(1, S);
    for i = 1:S
        etiquetas{i} = ['b_' num2str(i)];
    end;
    legend(etiquetas);
    title('Valores del bias');
end
        \end{lstlisting}