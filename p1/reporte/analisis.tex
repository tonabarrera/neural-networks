\section{Discusión de resultados}
\subsection{Hamming}
Es fácil darse cuenta que los resultados fueron correctos debido a la simplicidad de esta red, estos pueden ser comprobados fácilmente por uno mismo las únicas posibles diferencias que pueden existir son al elegir los valores que son aleatorios como lo son en este caso épsilon. Esta variable de la red podría producir que la red termine de trabajar en un mayor o menor numero de iteraciones.

Otro punto importante a mencionar es el como el funcionamiento de la red es muy evidente en las gráficas de la evolución de sus salidas (véanse figuras \ref{fig:hamming1}, \ref{fig:hamming2}, \ref{fig:hamming3}). Donde como era de esperarse la neurona con el valor inicial más alto es la que disminuye más lento mientras que el resto lo hace con una pendiente más pronunciada. Esto se debe a que el vector de entrada tiene más valores en común con algunos vectores que con otros (menor distancia de Hamming), de nuevo esto se ve reflejado en la gráfica donde a mayor numero de puntos en común mayor valor inicial en la capa recurrente.

Por ultimo, aquellos vectores con misma distancia de Hamming con respecto al vector de entrada tienen el mismo comportamiento en la gráfica.
\subsection{Perceptron}
\subsection{ADALINE}
Respecto a los experimentos realizados en ADALINE sin bias podemos observar que entre mayor sea el tamaño del codificador a trabajar a la red le toma más iteraciones converger a un resultado. Además por las gráficas \ref{fig:adaline3error} y \ref{fig:adaline4error} podemos observar que el error tiende a disminuir bastante rápido en las primeras iteraciones para después pasar a hacerlo de una forma más lenta hasta alcanzar un valor aceptable.

De igual forma los pesos en estos dos experimentos crecen a un ritmo similar en las primeras iteraciones para después tender a estabilizarse en valores diferentes, para este punto no hay un comportamiento claro ya que algunos valores empiezan a crecer mientras otros decrecen pero algo que tienen en común es que en las ultimas iteraciones estos ya no sufren demasiados cambios lo cual es obvio por la gráfica de la evolución del error.

En ADALINE con bias la historia se repite ya que los comportamientos de las gráficas son bastante similares a su contraparte sin bias, suceden cambios drásticos en las primeras iteraciones para después proceder a hacer pequeños ajustes en las ultimas iteraciones.