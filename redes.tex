\documentclass{article}
\usepackage[utf8]{inputenc}
\usepackage[letterpaper, margin=2.5cm]{geometry}
\usepackage[spanish]{babel}
\begin{document}
A pesar de todos los avances científicos actuales el funcionamiento de las redes neuronales aun es un tema no muy explorado, aun así se sabe que juegan un papel importante en el aprendizaje ya que este se entiende como la creaciones de conexiones entre neuronas o la modificación de conexiones ya existentes. Con esto claro se puede entender el porque es importante simular este tipo de comportamientos mediante la creación de neuronas artificiales y el aprendizaje que estas realicen.

Al igual que cualquier tecnología, las redes neuronales necesitan dos elementos para que se puedan desarrollar los cuales son un concepto y la implementación. El primero hace referencia a el hecho de tener una idea una forma de pensamiento sobre algún tema que antes no existía. Por otro lado, la implementación es necesaria para poder hacer realidad el concepto que ya existe de lo contrario la tecnología no podrá madurar.

La historia de las redes neuronales tuvo altibajos, su inicio esta presente al inicio del siglo veinte con trabajos en la física, psicología y neurosicologia que surgieron de científicos como Hermann von Helmhotz, Ernst Mach e Ivan Pavlov. En los años cuarenta fue cuando comenzó una visión más moderna de las redes neuronales que se base en que las redes neuronales pueden computar cualquier función aritmética o lógica concepto formulado por Warren McCulloch y Walter Pitts.

Por otro lado, la aplicación de las redes neuronales apareció en los cincuenta con la creación de el perceptron y las reglas de aprendizaje de parte de Frank Rosenblatt. Otro algoritmo de aprendizaje importante fue el Widrow-Hoff el cual se sigue usando hoy en día. 

Sin embargo, estos algoritmos son muy simples y no sirven para entrenar a redes más complejas. Esto y el hecho de que no existían computadoras muy poderosas genero la idea de que las redes neuronales habían llegado a su fin. Pocos aportes se realizaron durante este periodo como es el caso de redes neuronales que actuaban como memoria desarrolladas por Teuvo Kohonen y James Anderson y redes autoorganizables creadas por Stephen Grossberg.

En los años ochenta las redes neuronales volvieron a ser importantes, y se hicieron aportes como el uso de mecanismos estadisticos para explicar el comportamiento de una red recurrente para ser usada como memoria y el algoritmo para entrenar entrenar varias redes perceptron que fue descubierto de manera individual por diferentes investigadores.

Al mismo tiempo que surgían nuevos conceptos surgían nuevas implementaciones para las redes neuronales, entre las implementaciones más notables se encuentran:
\begin{itemize}
	\item Pilotos automaticos para aviones y autos
	\item Simulaciones de vuelo
	\item Control de inyección de combustible
	\item Reconocimiento Facial
	\item Identificación de particulas en tiempo real
	\item Reconocimiento de voz
	\item Traducción en tiempo real de un lenguaje hablado
\end{itemize}

\end{document}