\section{Introducción}
El objetivo del siguiente trabajo es sentar las bases de los principales puntos 
en el estudio del Perceptrón Multicapa (Multilayer Perceptron) entre los que 
están sus características, las partes que lo conforman y el como estas partes 
trabajan juntas para lograr el funcionamiento que el MLP presenta. Conociendo 
su funcionamiento se puede determinar cuales son las principales actividades en 
las que es empleada esta arquitectura de redes neuronales lo cual, a su vez, 
explica el porque dicha red es de tal importancia en el campo de las redes 
neuronales.
\\\\
Sin embargo, todo este conocimiento teórico seria nada si no se ve aplicado a 
algún problema en especifico, es debido a esto que en esta práctica se empleo 
el perceptrón multicapa para realizar la aproximación de señales (la cual es 
una de las principales aplicaciones que tiene esta red) esta aproximación fue 
desarrollada utilizando la herramienta MATLAB, entre las principales 
características que tiene el programa desarrollado están.
\begin{itemize}
 \item Entrada de datos por parte del usuario.
 \item Distintos métodos para determinar la convergencia de la red.
 \item Graficación de los resultados obtenidos por el perceptrón.
\end{itemize}
Aunado a esta implementación se encuentra la discusión de resultados en la cual 
se realiza un análisis de los datos obtenidos de los resultados 
experimentales con distintos casos de pruebas. Al hacer dicho análisis se llego 
a diferentes conclusiones respecto a la implementación realizada y en generar a 
la red neuronal utilizada como es el caso de sus ventajas y las limitaciones 
que tiene el uso de esta.