\section{Conclusiones}
Sin duda alguna el perceptrón multicapa es una herramienta sumamente poderosa no por nada es la arquitectura más utilizada actualmente, esta práctica solo es una muestra de ello debido a que solo se utilizo para la aproximación de funciones. Sin embargo, es importante tener en cuenta que para su correcto funcionamiento se deben considerar diversos factores.

Dichos factores afectaron a la elaboración de esta práctica como el hecho de no tener tantos datos como ocurrió en el primer experimento realizado. Otra cuestión que es importante mencionar es que el tiempo de ejecución puede llegar a ser bastante alto por lo que contar con un buen hardware es de suma importancia, en este caso los tiempos alcanzaron casi 10 minutos para la máxima cantidad de datos y 5000 iteraciones. Lo cual puede parecer poco tiempo pero si se tratara de usar en un ambiente real se puede llegar a trabajar con millones de datos por lo cual no seria de utilidad, dicho problema se podría solucionar al utilizar otro lenguaje de programación más rápido como es el caso de C.

Finalmente, es evidente que se puede mejorar el entrenamiento utilizando valores de pesos, bias y factor de aprendizaje más específicos por lo que seria de gran utilidad tener un pre procesamiento de los datos que genere mejores valores de dichos parámetros y no solo utilizar valores aleatorios.